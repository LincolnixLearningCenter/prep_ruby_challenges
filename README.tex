% Created 2016-04-18 Mon 11:38
\documentclass[11pt]{article}
\usepackage[utf8]{inputenc}
\usepackage[T1]{fontenc}
\usepackage{fixltx2e}
\usepackage{graphicx}
\usepackage{longtable}
\usepackage{float}
\usepackage{wrapfig}
\usepackage{rotating}
\usepackage[normalem]{ulem}
\usepackage{amsmath}
\usepackage{textcomp}
\usepackage{marvosym}
\usepackage{wasysym}
\usepackage{amssymb}
\usepackage{hyperref}
\tolerance=1000
\date{\today}
\title{README}
\hypersetup{
  pdfkeywords={},
  pdfsubject={},
  pdfcreator={Emacs 24.5.1 (Org mode 8.2.10)}}
\begin{document}

\maketitle
\textbf{prep$_{\text{ruby}}$$_{\text{challenges}}$}

\href{http://www.vikingcodeschool.com/web-markup-and-coding/level-up-your-ruby-judo}{The Ruby challenge problems from the Markup and Coding course of the Viking Code School Prep Work}

\url{https://www.vikingcodeschool.com}

\section{Ruby Calisthenics}
\label{sec-1}

\subsection{Power}
\label{sec-1-1}

Write a method \emph{power} which takes two integers (\emph{base} and \emph{exponent}) and 
returns the \emph{base} raised to the power of \emph{exponent}. Do not use Ruby's "**"
operator for this!

\begin{verbatim}
> power(3,4)
=> 81 # (3*3*3*3)
\end{verbatim}

\begin{verbatim}
def power(base,exponent)
  # returns base raised to the power of exponent without the use of ** operator

  a = base
  b = exponent
  c = []

  b.times do
    c.push a
  end

  c.inject(1) {|product, n| product * n}
end

p power(3,4)
\end{verbatim}

\subsection{Factorial}
\label{sec-1-2}

Write a method \emph{factorial} which takes a number and returns the product of 
every number up to the current number multiplied together.

\begin{verbatim}
> factorial(5)
=> 120 # from 1*2*3*4*5
\end{verbatim}

\begin{verbatim}
def factorial(n)
  # Int => Int
  # Takes a number and returns the product of every number up to 
  # the current number multiplied together

  a = []

  n.downto(1).each do |i|
    a.push i
  end

  return a.inject(1) {|product, n| product * n}

end

p factorial(5)
\end{verbatim}

\subsection{Uniques}
\label{sec-1-3}

Write a method \emph{uniques} which takes an array of items and returns the array
without any duplicates. Don't use Ruby's \emph{uniq} method.

\begin{verbatim}
uniques([1,5,"frog",2,1,3,"frog"])
=> [1,5,"frog",2,3]
\end{verbatim}

\begin{verbatim}
def uniques(array)
  # Array of Items => Array of Items
  # Takes an array, returns array with duplicate items removed.
  # Write without uniq

  no_dupes = []
  couples = array.combination(2)
  groups = array.group_by{|e| e}

  groups.each do |g|
    no_dupes.push(g[0])
  end

  return no_dupes
end

p uniques([1,5,"frog",2,1,3,"frog"])
\end{verbatim}

\subsection{Combinations}
\label{sec-1-4}

Write a method \emph{combinations} which takes two arrays of strings and returns
an array with all of the combinations of the items in them, listing the first
items first.

\begin{verbatim}
> combinations(["on","in"],["to","rope"])
=> ["onto","onrope","into","inrope"]
\end{verbatim}

\begin{verbatim}
def combinations(ary1,ary2)
  # Ary(Str), Ary(Str) => Ary(Str)
  # Takes two arrays of strings, returns an array with all of the combinations
  # of the items in them, listing the first item first.

  a = ary1
  b = ary2

  c = []

  a.each do |s|
    b.each do |x|
      c.push "#{s}#{x}"
    end
  end

  p c
end

combinations(["on","in"],["to","rope"])
\end{verbatim}

\subsection{Primes}
\label{sec-1-5}

Write a method \emph{is$_{\text{prime}}$?} which takes in a number and returns \emph{true} if it 
is a prime number.

\begin{verbatim}
> is_prime?(7)
=> true
> is_prime?(14)
=> false
\end{verbatim}

\begin{verbatim}
def is_prime?(i)
  range = (i-1).downto(2)

  range.each do |a|
    #p i%a == 0
  end

  p range.any? {|a| i%a == 0}
end

is_prime?(7)
\end{verbatim}

\subsection{Rectangle Overlap}
\label{sec-1-6}

Write a method \emph{overlap} which takes two rectangles defined by the 
coordinates of their corners, e.g. \emph{[[0,0],[3,3]]} and \emph{[[1,1],[4,6]]},
and determines whether they overlap. You can assume all coordinates are 
positive integers.

\begin{verbatim}
> overlap( [ [0,0],[3,3] ], [ [1,1],[4,5] ] )
=> true
> overlap( [ [0,0],[1,4] ], [ [1,1],[3,2] ] )
=> false 
\end{verbatim}

It doesn't count as overlapping if their edges touch but they do not 
otherwise overwrite each other. As expressed by a sixth grade student:

\includegraphics[width=.9\linewidth]{./coordinate_overlaps.png}

\begin{verbatim}
def overlap(a,b)
  # Array(Coordinates), Array(Coordinates) => Boolean

  # a = [[0,0],[3,3]]
  ax1 = a[0][0]
  ay1 = a[0][1] 
  ax2 = a[1][0]
  ay2 = a[1][1]

  awidth = ax2-ax1
  aheight = ay2-ay1
  aarea = awidth*aheight

  # b = [[1,1],[4,5]]
  bx1 = b[0][0]
  by1 = b[0][1]
  bx2 = b[1][0]
  by2 = b[1][1]

  bwidth = bx2-bx1
  bheight = by2-by1
  barea = bwidth*bheight

  #( [ [0  , 0  ],[3  , 3  ] ], [ [1  , 1  ],[4  , 5  ] ] )
  #( [ [ax1, ay1],[ax2, ay2] ], [ [bx1, by1],[bx2, by2] ] )

  if bx1 < ax2 && by1 < ay2
    true
  end
end

overlap( [ [0,0],[3,3] ], [ [1,1],[4,5] ] )
overlap( [ [0,0],[1,4] ], [ [1,1],[3,2] ] )

# further development needed to explore every case
\end{verbatim}

\section{A Bigger Challenge: The Counting Game}
\label{sec-2}

10 friends are sitting in a circle around a table and decide to play a new 
game. In it, they count up through the numbers from 1 to 100. The first person
says "1", the second says "2" and so on\ldots{} but with a few catches:

\begin{itemize}
\item Whenever the number is divisible by 7, they switch directions. So person 6 
will say "6", person 7 will say "7", then person 6 again will say "8".

\begin{verbatim}
when x%y == 0 # reverse
\end{verbatim}

\item Whenever the number is divisible by 11, they skip the next person for the 
following number. For instance, if person 3 says "33", person 5 will say 
"34" instead (person 4 gets skipped).

\begin{verbatim}
friends = []
10.times do 
  friends.push 0
end
\end{verbatim}

\begin{verbatim}
# Produces each number and which person said it
# Hash {Person(Int)=>List of Numbers(Array of Integers)}

# Example Return Steps
# { 1 => 1, 2 => 2, 3 => 3, 4 => 4, 5 => 5, 6 => 6, 7 => 7, 8 => nil, 9 => nil, 10 => nil }
# { 1 => [1,12], 2 => 2, 3 => [3,11], 4 => [4,10], 5 => [5,9], 6 => [6,8], 7 => 7, 8 => nil, 9 => nil, 10 => nil }
# { 1 => [1,12], 2 => 2, 3 => [3,11], 4 => [4,10], 5 => [5,9], 6 => [6,8], 7 => 7, 8 => nil, 9 => 14, 10 => [13,15] }
# { 1 => [1,12,16,25], 2 => [2,17,24], 3 => [3,11,18,23], 4 => [4,10,19], 5 => [5,9,20,22], 6 => [6,8,21], 7 => 7, 8 => nil, 9 => 14, 10 => [13,15] }
\end{verbatim}
\end{itemize}

\subsection{The Elevator}
\label{sec-2-1}

You live in a 25 story building with one elevator. The central 
microcontroller got eaten by rats and the building manager has asked you to 
code up the elevator's operating procedure until he can get a new one. You 
figure you'll have to learn to actually code soon but you first want to think
things through and pseudocode your design.

\subsubsection{Elevator Details}
\label{sec-2-1-1}

The basic elevator machinery is completely dumb (it doesn't do anything it's
not told to do) but is capable of interpreting and executing the commands:

\begin{itemize}
\item "open elevator door"
\item "close elevator door"
\item "go up full speed"
\item "go down full speed"
\item "slow down"
\item "stop"
\end{itemize}


\ldots{}and it accepts user input in the form of:

\begin{itemize}
\item floor buttons inside the elevator
\item door open and close buttons inside the elevator
\item up and down call buttons on each floor
\end{itemize}


\ldots{}and it has sensors for:

\begin{itemize}
\item if a human is in the door closing path
\item if it is currently at a floor (instead of in-between floors)
\end{itemize}


\ldots{}and it has a few quirky requirements:

\begin{itemize}
\item it must "slow down" at least 1 floor before it stops.
\item there is a small chance that it actually stops between floors by 
accident (it's an old elevator)
\end{itemize}

\subsubsection{The Task}
\label{sec-2-1-2}

Your job is to design a properly working elevator. It should stop on each 
floor it is physically able to during a given trip to pick up whoever is 
going the same direction. Additionally, make sure that no one is:

\begin{enumerate}
\item smashed into the ground
\item pushed through the roof
\item squished by the doors
\item let off in between floors
\item stuck going the wrong direction (unless they choose not to exit)
\end{enumerate}


This will be good practice thinking about all the edge cases and scenarios 
that a user can do.

The point isn't to follow any strict guidelines of syntax but rather to 
focus on getting the logic of the problem figured out and then organizing it
into modules that accomplish the sub-tasks that are required.


\subsection{NB: Software Engineering}
\label{sec-2-2}

\url{https://www.vikingcodeschool.com/software-engineering-basics}

\begin{itemize}
\item "logic" way through problems
\begin{itemize}
\item pseudocoding ("whiteboarding")
\item modular design and engineering best practices
\item 4-step engineering problem solving approach
\begin{enumerate}
\item Understand the problem
\item Plan a solution
\item Carry out that plan
\item Examine your results for accuracy
\end{enumerate}
\item Agile development
\begin{itemize}
\item project management technique / development philosophy
\item teams commonly work in short (1-2 week) sprints
\item XP and SCRUM, Agile techniques
\begin{itemize}
\item short cycle times
\item frequent client/user interaction
\begin{itemize}
\item keeps project focused on relevant tasks
\end{itemize}
\item XP
\begin{itemize}
\item pair programming
\begin{itemize}
\item pairing developers together at workstations
\end{itemize}
\end{itemize}
\end{itemize}
\item keep software user-driven
\item TDD
\end{itemize}
\end{itemize}
\end{itemize}
% Emacs 24.5.1 (Org mode 8.2.10)
\end{document}
