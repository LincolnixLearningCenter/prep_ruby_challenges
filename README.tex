% Created 2016-04-16 Sat 04:06
\documentclass[11pt]{article}
\usepackage[utf8]{inputenc}
\usepackage[T1]{fontenc}
\usepackage{fixltx2e}
\usepackage{graphicx}
\usepackage{longtable}
\usepackage{float}
\usepackage{wrapfig}
\usepackage{rotating}
\usepackage[normalem]{ulem}
\usepackage{amsmath}
\usepackage{textcomp}
\usepackage{marvosym}
\usepackage{wasysym}
\usepackage{amssymb}
\usepackage{hyperref}
\tolerance=1000
\date{\today}
\title{README}
\hypersetup{
  pdfkeywords={},
  pdfsubject={},
  pdfcreator={Emacs 24.5.1 (Org mode 8.2.10)}}
\begin{document}

\maketitle
\textbf{prep$_{\text{ruby}}$$_{\text{challenges}}$}

\href{http://www.vikingcodeschool.com/web-markup-and-coding/level-up-your-ruby-judo}{The Ruby challenge problems from the Markup and Coding course of the Viking Code School Prep Work}

\section{Ruby Calisthenics}
\label{sec-1}

\subsection{Power}
\label{sec-1-1}

Write a method \emph{power} which takes two integers (\emph{base} and \emph{exponent}) and 
returns the \emph{base} raised to the power of \emph{exponent}. Do not use Ruby's \emph{**}
operator for this!

\begin{verbatim}
> power(3,4)
=> 81 # (3*3*3*3)
\end{verbatim}

\subsection{Factorial}
\label{sec-1-2}
\subsection{Uniques}
\label{sec-1-3}
\subsection{Combinations}
\label{sec-1-4}
\subsection{Primes}
\label{sec-1-5}
\subsection{Rectangle Overlap}
\label{sec-1-6}
% Emacs 24.5.1 (Org mode 8.2.10)
\end{document}
